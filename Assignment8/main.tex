\documentclass[journal,12pt,twocolumn]{IEEEtran}

\usepackage{setspace}
\usepackage{gensymb}
\singlespacing
\usepackage[cmex10]{amsmath}

\usepackage{amsthm}

\usepackage{mathrsfs}
\usepackage{txfonts}
\usepackage{stfloats}
\usepackage{bm}
\usepackage{cite}
\usepackage{cases}
\usepackage{subfig}

\usepackage{longtable}
\usepackage{multirow}

\usepackage{enumitem}
\usepackage{mathtools}
\usepackage{steinmetz}
\usepackage{tikz}
\usepackage{circuitikz}
\usepackage{verbatim}
\usepackage{tfrupee}
\usepackage[breaklinks=true]{hyperref}
\usepackage{graphicx}
\usepackage{tkz-euclide}

\usetikzlibrary{calc,math}
\usepackage{listings}
    \usepackage{color}                                            %%
    \usepackage{array}                                            %%
    \usepackage{longtable}                                        %%
    \usepackage{calc}                                             %%
    \usepackage{multirow}                                         %%
    \usepackage{hhline}                                           %%
    \usepackage{ifthen}                                           %%
    \usepackage{lscape}     
\usepackage{multicol}
\usepackage{chngcntr}

\DeclareMathOperator*{\Res}{Res}

\renewcommand\thesection{\arabic{section}}
\renewcommand\thesubsection{\thesection.\arabic{subsection}}
\renewcommand\thesubsubsection{\thesubsection.\arabic{subsubsection}}

\renewcommand\thesectiondis{\arabic{section}}
\renewcommand\thesubsectiondis{\thesectiondis.\arabic{subsection}}
\renewcommand\thesubsubsectiondis{\thesubsectiondis.\arabic{subsubsection}}


\hyphenation{op-tical net-works semi-conduc-tor}
\def\inputGnumericTable{}                                 %%

\lstset{
%language=C,
frame=single, 
breaklines=true,
columns=fullflexible
}
\begin{document}


\newtheorem{theorem}{Theorem}[section]
\newtheorem{problem}{Problem}
\newtheorem{proposition}{Proposition}[section]
\newtheorem{lemma}{Lemma}[section]
\newtheorem{corollary}[theorem]{Corollary}
\newtheorem{example}{Example}[section]
\newtheorem{definition}[problem]{Definition}

\newcommand{\BEQA}{\begin{eqnarray}}
\newcommand{\EEQA}{\end{eqnarray}}
\newcommand{\define}{\stackrel{\triangle}{=}}
\bibliographystyle{IEEEtran}
\raggedbottom
\setlength{\parindent}{0pt}
\providecommand{\mbf}{\mathbf}
\providecommand{\pr}[1]{\ensuremath{\Pr\left(#1\right)}}
\providecommand{\qfunc}[1]{\ensuremath{Q\left(#1\right)}}
\providecommand{\sbrak}[1]{\ensuremath{{}\left[#1\right]}}
\providecommand{\lsbrak}[1]{\ensuremath{{}\left[#1\right.}}
\providecommand{\rsbrak}[1]{\ensuremath{{}\left.#1\right]}}
\providecommand{\brak}[1]{\ensuremath{\left(#1\right)}}
\providecommand{\lbrak}[1]{\ensuremath{\left(#1\right.}}
\providecommand{\rbrak}[1]{\ensuremath{\left.#1\right)}}
\providecommand{\cbrak}[1]{\ensuremath{\left\{#1\right\}}}
\providecommand{\lcbrak}[1]{\ensuremath{\left\{#1\right.}}
\providecommand{\rcbrak}[1]{\ensuremath{\left.#1\right\}}}
\theoremstyle{remark}
\newtheorem{rem}{Remark}
\newcommand{\sgn}{\mathop{\mathrm{sgn}}}
\providecommand{\abs}[1]{\left\vert#1\right\vert}
\providecommand{\res}[1]{\Res\displaylimits_{#1}} 
\providecommand{\norm}[1]{\left\lVert#1\right\rVert}
%\providecommand{\norm}[1]{\lVert#1\rVert}
\providecommand{\mtx}[1]{\mathbf{#1}}
\providecommand{\mean}[1]{E\left[ #1 \right]}
\providecommand{\fourier}{\overset{\mathcal{F}}{ \rightleftharpoons}}
%\providecommand{\hilbert}{\overset{\mathcal{H}}{ \rightleftharpoons}}
\providecommand{\system}{\overset{\mathcal{H}}{ \longleftrightarrow}}
	%\newcommand{\solution}[2]{\textbf{Solution:}{#1}}
\newcommand{\solution}{\noindent \textbf{Solution: }}
\newcommand{\cosec}{\,\text{cosec}\,}
\providecommand{\dec}[2]{\ensuremath{\overset{#1}{\underset{#2}{\gtrless}}}}
\newcommand{\myvec}[1]{\ensuremath{\begin{pmatrix}#1\end{pmatrix}}}
\newcommand{\mydet}[1]{\ensuremath{}}}
\newcommand{\comb}[2]{{}^{#1}\mathrm{C}_{#2}}
\numberwithin{equation}{subsection}

\makeatletter
\@addtoreset{figure}{problem}
\makeatother
\let\StandardTheFigure\thefigure
\let\vec\mathbf

\renewcommand{\thefigure}{\theproblem}

\def\putbox#1#2#3{\makebox[0in][l]{\makebox[#1][l]{}\raisebox{\baselineskip}[0in][0in]{\raisebox{#2}[0in][0in]{#3}}}}
     \def\rightbox#1{\makebox[0in][r]{#1}}
     \def\centbox#1{\makebox[0in]{#1}}
     \def\topbox#1{\raisebox{-\baselineskip}[0in][0in]{#1}}
     \def\midbox#1{\raisebox{-0.5\baselineskip}[0in][0in]{#1}}
\vspace{3cm}
\title{Assignment 8}
\author{Tanmay Goyal - AI20BTECH11021}
\maketitle
\newpage
\bigskip
\renewcommand{\thefigure}{\theenumi}
\renewcommand{\thetable}{\theenumi}

%
Download all latex-tikz codes from 
%
\begin{lstlisting}
https://github.com/tanmaygoyal258/AI1103---Probability/blob/main/Assignment8/main.tex
\end{lstlisting}
\section{Problem}
A fair coin is tossed repeatedly. Let $X$ be the number of tails before the first heads occurs. Let $Y$ denote the number of tails between the first and second heads. Let $X+Y = N$. Then which of the following are true?\\
\begin{enumerate}
    \item X and Y are independent random variables with
    {\footnotesize
    \begin{align}
        \pr{X = k} = \pr{Y = k} =
        \begin{cases}
            2^{-(k+1)} & k=0,1,2 \ldots
            \\
            0 & otherwise
        \end{cases}
    \end{align}
    }
    \item $N$ has a probability mass function given by
    {\small
     \begin{align}
        \pr{N = k} =
        \begin{cases}
            (k-1)2^{-k} & k=2,3,4 \ldots
            \\
            0 & otherwise
        \end{cases}
    \end{align}
    }
    \item Given $N = n$, the conditional distribution of X and Y are independent

    \item Given $N = n$
     \begin{align}
        \pr{X = k} =
        \begin{cases}
            \frac{1}{n+1} & n=0,1,2 \ldots
            \\
            0 & otherwise
        \end{cases}
    \end{align}
\end{enumerate}

\section{Solution}
Let us consider Tails as a failure and Heads as a success, then $X$ and $Y$, both, can be seen to be \underline{geometric random variables}, with $p = \frac{1}{2}$ and:
\begin{align}
    \pr{X = k} = \pr{Y = k} = (1-p)^k p  
    \label{formula}
\end{align}
\begin{enumerate}
    \item \underline{Option 1:}\\
    
    Substituting the value of $p$ in \eqref{formula}:
{\small
\begin{align}
    \pr{X = k} = \pr{Y = k} = \frac{1}{2^{k+1}} = 2^{-(k+1)}
\end{align}
}
To test for independence of $X$ and $Y$, we calculate $\pr{X = k, Y = k}$, which means obtaining $k$ tails, 1 head, $k$ tails, and one head, in order. Thus,
{\small
\begin{align}
    \pr{X = k,Y = k} = (1-p)^k p \times (1-p)^k p\\
    =\pr{X = k}\pr{Y = k}
    \label{independence}
    \end{align}
    }
    Thus, $X$ and $Y$ are independent, and hence, \textbf{Option 1 is correct}\\

    \item \underline{Option 2:}\\
    
    From \eqref{formula} and \eqref{independence}, we get:
\begin{align}
    \pr{N = k} = \pr{X+Y = k}\\
    = \sum_{i=0}^k\pr{X = i, Y = k-i} \\
    = \sum_{i=0}^k\pr{X = i} \pr{Y = k-i}\\
    =\sum_{i=0}^k (1-p)^i p (1-p)^{k-i} p\\
    = p^2(1-p)^k (k+1)\\
    =(k+1) 2^{-(k+2)}
\end{align}
Hence, \textbf{Option 2 is incorrect}\\

 \item \underline{Option 3:}\\
 
 We know, if a conditional distribution is independent, then:
\begin{align}
    \pr{X = x,Y=y | Y=y} = \pr{X=x}
\end{align}
Thus, the conditional distribution of $X$ given $N=n$:\\

{\small
\begin{align}
    \pr{X=k|N=n} = \frac{\pr{X=k,X+Y=n}}{\pr{N=n}}\\
    =\frac{\pr{X=k,Y=n-k}}{\pr{N=n}}\\
    =\frac{2^{-(k+1)}2^{-(n-k+1)}}{(n+1) 2^{-(n+2)}}\\
    = \frac{1}{n+1} 
    \label{option4}\\
    \neq \pr{X = k}
\end{align}
}
Similarly,
\begin{align}
    \pr{Y=k|N=n} = \frac{1}{n+1}\\
    \neq \pr{Y=k}
\end{align}
Thus, the condition for independence fails and hence, \textbf{Option 3 is incorrect}\\

\item \underline{Option 4:}\\

From \eqref{option4}, we see that \textbf{Option 4 is correct}\\
    
\end{enumerate}

The correct options are \textbf{(1)} and \textbf{(4)}
\end{document}
